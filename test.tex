\documentclass[]{NUKThesis}

% For Lorem lipsum
\usepackage{lipsum}
\usepackage{zhlipsum}

% For equation
\usepackage{mathtools}

\title{中文標題}
\etitle{English title}
\author{學生}
\professor{老師}
\time{中華民國一零九年四月}

\begin{document}
  \coverpage
  \pagenumbering{roman}
  \tableofcontents
  \clearpage
  \listoffigures
  \clearpage
  \listoftables
  \clearpage

  {\noindent \Large 致謝}

  \zhlipsum[4-5][name=nanshanjing]
  \clearpage
  \pagenumbering{arabic} % Reset the page number counter
  
  % =====================================
  % |   Your Thesis Starts From Here    |
  % =====================================
  \section{介紹}

  \zhlipsum[1-4][name=nanshanjing]

  \begin{equation}
    X(\omega) = 
      \begin{dcases*}
       1 & $\omega\in A$\\
       0 & $\omega \in A^c$
      \end{dcases*}
  \end{equation}

  \section{各種圖和表}

  \zhlipsum[5][name=nanshanjing]
  \begin{figure}[h]
    \centering
    \includegraphics[width=8cm]{figure}
    \caption{當你 compile 一次就過}
  \end{figure}

  \zhlipsum[6-8][name=nanshanjing]
  
  \begin{figure}[h]
    \centering
    \includegraphics[width=8cm]{figure2}
    \caption{嚶嚶嚶嚶嚶}
  \end{figure}
  
  \zhlipsum[8-10][name=nanshanjing]
  
  \begin{table}[h]
    \centering
    \begin{tabular}{|c|c|c|} 
      \hline
      cell1 dummy text dummy text dummy text& cell2 & cell3 \\ 
      \hline
      \hline
      cell1 dummy text dummy text dummy text & cell5 & cell6 \\ 
      \hline
      cell7 & cell8 & cell9 \\ 
      \hline
    \end{tabular}
    \caption{就是一個表格}
  \end{table}
  
  \zhlipsum[9-10][name=nanshanjing]

\end{document}